\documentclass{article}
\usepackage{graphicx}
\graphicspath{ {./images/} }
\usepackage[export]{adjustbox}
\usepackage{amsmath}


\title{1. matematiskas loģikas majadarbs}
\author{Gunārs Ābeltiņš}
\date{2024-03-08}

\begin{document}

\maketitle

\section{Formālās teorijas}

\subsection*{a apakšuzdevums}

\begin{enumerate}
    \item Valoda: {b, c, d}
    \item Aksiomas: b, c
    \item Izveduma likumi: $x \vdash dxd$, $x \vdash xbcb$
    \item Teorēmas
          \[
              b \vdash \underline{d}b\underline{d} \vdash dbd\underline{bcb} \vdash \underline{d}dbdbcb\underline{d} \vdash \underline{d}ddbdbcbd\underline{d}
          \]
          \[
              c \vdash \underline{d}c\underline{d} \vdash dcd\underline{bcb} \vdash \underline{d}dcdcb\underline{d} \vdash \underline{d}ddcdbcbd\underline{d}
          \]
\end{enumerate}

\subsection*{b apakšuzdevums}

\subsubsection*{Nepierādami apgalvojumi}

\begin{enumerate}
    \item[$d$:] Vārds satur tikai vienu simbolu un tas nav aksionma
    \item[$bcd$:] Vārdu nevar izvest izmantojot izveduma likumus
\end{enumerate}

\subsubsection*{Algoritms}

\begin{enumerate}
    \item Ja vārda ir tikai viens simbols:
          \begin{enumerate}
              \item Ja simbols ir aksioma, tad apgalvojums ir pierādāms
              \item Ja nav aksioma, tad apgalvojums nav pierādāms
          \end{enumerate}
    \item Ja vārds beidzās ar "bcb", tad to daļu noņem un atgriežas pie pirmā soļa.
    \item Ja vārds sākas un beidzās ar "d", tad šīs daļas noņem un atgriežas pie pirmā soļa.
    \item Ja ir nokļuvis līdz šim solim, tad vārds nav pierādāms.
\end{enumerate}

\pagebreak

\section{Predikātu valodas}

Predikātu valodā C ir šādi 5 predikāti: S(x) un V(x) (x ir sieviete/vīrietis), T(x, y), M(x, y)
(x ir y-a (bioloģiskais) tēvs/māte), P(x,y) (x ir precējies ar y), kā arī predikāts x=y.

\subsection*{1. apakšuzdevums}

\begin{enumerate}
    \item x ir vectēvs:
          \[
              \exists y \exists z (\neg (x = y) \land T(x, y) \land (T(y, z) \lor M(y,z)))
          \]

    \item x un y ir brāļi:
          \[
              V(x) \land V(y) \land \exists z (T(z, x) \land T(z, y) \lor M(z, x) \land M(z, y))
          \]

    \item x-am visi bērni ir dēli:
          \[
              \forall y ((T(x,y) \lor M(x, y)) \rightarrow V(y))
          \]

    \item x ir divu meitu tēvs:
          \[
              V(x) \land \exists y \exists z (\neg (y = z) \land S(y) \land S(z) \land T(x, y) \land T(x, z))
          \]
\end{enumerate}

\subsection*{2. apakšuzdevums}

\begin{enumerate}
    \item x-a māte un tēvs ir precējušies:
          \[
              \forall y \forall z (\neg(y = z) \land (M(x, y) \land T(x, z)) \rightarrow P(y, z))
          \]

    \item Precēties var tikai dažādi dzimumi:
          \[
              \forall x \forall y ((P(x, y) \land V(x)) \rightarrow S(y))
          \]

    \item x ir y-a māsīca:
          \begin{align*}
               & S(x) \land \exists z \exists v \exists w                    \\
               & \neg (z = v \lor z = w \lor v = w)                          \\
               & \land ((M(z, v) \lor T(z, v)) \land (M(z, w) \lor T(z, w))  \\
               & \land (M(v, x) \lor T(v, x)) \land  (M(w, y) \lor T(w, y))) \\
          \end{align*}
\end{enumerate}

\pagebreak

\section*{3. Predikātu valodas}

\subsection*{1. apakšuzdevums}

Pierakstiet pirmās pakāpes aritmētikas valodā šādus apgalvojumus:

\begin{enumerate}
    \item[a)] x ir pirmskaitlis:
        \[
            P(x) = \neg (\exists y \exists z (x = y \cdot z \land \neg (y = x \lor z = x)))
        \]

    \item[b)] Katrs skaitlis, kas lielāks par 1, dalās ar kādu pirmskaitli:
        \[
            \forall x ((x > 1) \rightarrow \exists y \exists z (P(x) \land x = y \cdot z))
        \]

    \item[c)] x un y dalās ar vieniem un tiem pašiem pirmskaitļiem:
        \begin{align*}
             & \forall z (P(x) \rightarrow (\exists u (x = z \cdot u) \rightarrow                                  \\
             & \exists v (y = z \cdot v) \land (\exists v (y = z \cdot v) \rightarrow \exists u (x = z \cdot u))))
        \end{align*}
\end{enumerate}

\subsection*{2. apakšuzdevums}

Predikātu valodā S ir šādi 5 predikāti: Pasn(x) (x ir pasniedzējs), Stud(x) (x ir students),
Kurss(x) (x ir studiju kurss), pasniedz(x, y) (x pasniedz y), studē(x, y) (x studē y), ka arī
predikāts x=y.

\begin{enumerate}
    \item Kursi nav ne studenti, ne pasniedzēji:
          \[
              \forall x (Kurss(x) \rightarrow \neg (Stud(x) \lor Pasn(x)))
          \]

    \item Kursu $x$ pasniedz viens un tikai viens pasniedzējs:
          \[
              \forall y \forall z (Pasn(y) \land Pasn(z) \rightarrow (pasniedz(y, x) \land pasniedz(z, x) \rightarrow y = z))
          \]

    \item Pasniedzējs $x$ māca studentu $y$:
          \[
              Pasn(x) \land Stud(y) \land \exists z (pasniedz(x, z) \land stude(y, z))
          \]

    \item $x$ un $y$ ir studenti, kuri studē vismaz vienu kopīgu kursu:
          \[
              Stud(x) \land Stud(y) \land \exists z (stude(x, z) \land stude(y, z))
          \]
\end{enumerate}

\pagebreak

\section*{4. Pašam sava predikātu valoda.}

\subsection*{1. apakšuzdevums}

\begin{enumerate}
    \item Vērtību apgabals: aktieris, režisors, filma, skatītājs
    \item Objektu konstantes: nav
    \item Funkciju konstantes: nav
    \item Predikātu konstantes:
          \begin{itemize}
              \item $ aktieris(x) $ : $ x $ ir aktieris
              \item $ rezisors(x) $ : $ x $ ir režisors
              \item $ filma(x) $ : $ x $ ir filma
              \item $ skatitajs(x) $ : $ x $ ir skatītājs
              \item $ L(x,y) $: $ x $ ir lomā $ y $
              \item $ R(x,y) $: $ x $ režisē $ y $
              \item $ S(x,y) $: $ x $ ir noskatijies $ y $
          \end{itemize}
    \item Termi: objektu mainīgie
    \item Atomāras formulas: termi un predikātu konstantes
\end{enumerate}

\subsection*{2. apakšuzdevums}

\begin{enumerate}
    \item Katram aktierim ir vismaz viena loma:
          \[
              \forall x \exists y (aktieris(x) \rightarrow L(x, y))
          \]

    \item Skatītajs $x$ ir redzējis filmu ar aktieri $y$:
          \[
              skatitajs(x) \land aktieris(y) \land \exists z(filma(z) \land L(y, z) \land S(x, z))
          \]

    \item Skatītajs $x$ nav redzējis nevienu filmu ar režisoru $y$:
          \[
            skatitajs(x) \land rezisors(y) \land \forall z (filma(z) \land R(y, z) \rightarrow \neg S(x, z))
          \]
\end{enumerate}


\pagebreak
\section*{5. Aksiomas.}

\subsection*{1.4.2 b}

\[
    \begin{array}{l}
        [L_3 - L_5, \text{MP}]: A \land B \vdash B \land A \quad (8 \text{ formulas}) \\
        L_3: B \land C \rightarrow B                                                  \\
        L_4: B \land C \rightarrow C                                                  \\
        L_5: B \rightarrow (C \rightarrow B \land C)
    \end{array}
\]

\begin{enumerate}
    \item $A \land B$ \hfill (Dota hipotēze)
    \item $A \land B \rightarrow A$ \hfill (L$_3$)
    \item $A$ \hfill (MP 1, 2)
    \item $A \land B \rightarrow B$ \hfill (L$_4$)
    \item $B$ \hfill (MP 1, 4)
    \item $B \rightarrow (A \rightarrow B \land A)$ \hfill (L$_5$)
    \item $A \rightarrow B \land A$ \hfill (MP 5, 6)
    \item $B \land A$ \hfill (MP 3, 7)
\end{enumerate}

\subsection*{1.4.3 b}

\[
    \begin{array}{l}
        [L_3, L_4, L_9, \text{MP}]: \neg( A \land \neg A) \\
        L_3: B \land C \rightarrow B                      \\
        L_3: B \land C \rightarrow C                      \\
        L_9: (B \rightarrow C) \rightarrow ((B \rightarrow \neg C) \rightarrow \neg B)
    \end{array}
\]

\begin{enumerate}
    \item $A \land \neg A \rightarrow A$ \hfill (L$_3$)
    \item $A \land \neg A \rightarrow \neg A$ \hfill (L$_4$)
    \item $((A \land \neg A) \rightarrow A) \rightarrow (((A \land \neg A) \rightarrow \neg A) \rightarrow \neg(A \land \neg A))$ \hfill (L$_9$)
    \item $((A \land \neg A) \rightarrow \neg A) \rightarrow \neg(A \land \neg A)$ \hfill (MP 1, 3)
    \item $\neg(A \land \neg A)$ \hfill (MP 2, 4)
\end{enumerate}

\pagebreak
\section*{6. Formulu izvešana bez saīsinājumiem.}

\subsection*{1.4.2 c}

\[
    \begin{array}{l}
        [L_6 - L_8, \text{MP}]: A \lor B \rightarrow B \lor A \quad (5 \text{ formulas}) \\
        L_6: B \rightarrow B \lor C                                                      \\
        L_7: C \rightarrow B \lor C                                                      \\
        L_8: (B \rightarrow D) \rightarrow ((C \rightarrow D) \rightarrow (B \lor C \rightarrow D))
    \end{array}
\]

\begin{enumerate}
    \item $A \rightarrow B \lor A$ \hfill (L$_7$)
    \item $(A \rightarrow B \lor A) \rightarrow ((B \rightarrow B \lor A) \rightarrow (A \lor B \rightarrow B \lor A))$ \hfill (L$_8$)
    \item $(B \rightarrow B \lor A) \rightarrow (A \lor B \rightarrow B \lor A)$ \hfill (MP 1, 2)
    \item $B \rightarrow B \lor A$ \hfill (L$_6$)
    \item $A \lor B \rightarrow B \lor A$ \hfill (MP 3, 4)
\end{enumerate}

\subsection*{1.4.2 d}

\[
    \begin{array}{l}
        [L_1, L_9, \text{MP}]: B \land \neg B \vdash \neg C \quad (9 \text{ formulas}) \\
        L_1: B \rightarrow (C \rightarrow B)                                           \\
        L_9: (B \rightarrow C) \rightarrow ((B \rightarrow \neg C) \rightarrow \neg B)
    \end{array}
\]

\begin{enumerate}
    \item $B$ \hfill (Dota hipotēze)
    \item $\neg B$ \hfill (Dota hipotēze)
    \item $B \rightarrow (C \rightarrow B)$ \hfill (L$_1$)
    \item $\neg B \rightarrow (C \rightarrow \neg B)$ \hfill (L$_1$)
    \item $(C \rightarrow B)$ \hfill (MP 1, 3)
    \item $(C \rightarrow \neg B)$ \hfill (MP 2, 4)
    \item $(C \rightarrow B) \rightarrow ((C \rightarrow \neg B) \rightarrow \neg C)$ \hfill (L$_9$)
    \item $(C \rightarrow \neg B) \rightarrow \neg C$ \hfill (MP 5, 7)
    \item $\neg C$ \hfill (MP 6, 8)
\end{enumerate}

\pagebreak

\subsection{1.4.2 f}

\[
    \begin{array}{l}
        [L_1, L_8, L_{10}, \text{MP}]: \neg A \lor B \rightarrow (A \rightarrow B) \quad (5 \text{ formulas}) \\
        L_1: B \rightarrow (C \rightarrow B)                                                                  \\
        L_8: (B \rightarrow D) \rightarrow ((C \rightarrow D) \rightarrow (B \lor C \rightarrow D))           \\
        L_{10}: \neg B \rightarrow (B \rightarrow C)
    \end{array}
\]

\begin{enumerate}
    \item $\neg A \rightarrow (A \rightarrow B)$ \hfill (L$_{10}$)
    \item $B \rightarrow (A \rightarrow B)$ \hfill (L$_1$)
    \item $(\neg A \rightarrow (A \rightarrow B)) \rightarrow ((B \rightarrow (A \rightarrow B)) \rightarrow (\neg A \lor B \rightarrow (A \rightarrow B)))$ \hfill (L$_8$)
    \item $(B \rightarrow (A \rightarrow B)) \rightarrow (\neg A \lor B \rightarrow (A \rightarrow B))$ \hfill (MP 1, 3)
    \item $\neg A \lor B \rightarrow (A \rightarrow B)$ \hfill (MP 2, 4)
\end{enumerate}

\subsection{1.4.2 g}

\[
    \begin{array}{l}
        [L_8, L_{11}, \text{MP}]: A \rightarrow B, \neg A \rightarrow B \vdash B \quad (7 \text{ formulas}) \\
        L_8: (B \rightarrow D) \rightarrow ((C \rightarrow D) \rightarrow (B \lor C \rightarrow D))         \\
        L_{11}: B \lor \neg B
    \end{array}
\]

\begin{enumerate}
    \item $A \rightarrow B$ \hfill (Dotā hipotēze)
    \item $\neg A \rightarrow B$ \hfill (Dotā hipotēze)
    \item $A \lor \neg A$ \hfill (L$_{11}$)
    \item $(A \rightarrow B) \rightarrow ((\neg A \rightarrow B) \rightarrow (A \lor \neg A \rightarrow B))$ \hfill (L$_8$)
    \item $(\neg A \rightarrow B) \rightarrow (A \lor \neg A \rightarrow B)$ \hfill (MP 1, 4)
    \item $A \lor \neg A \rightarrow B$ \hfill (MP 2, 5)
    \item $B$ \hfill (MP 3, 6)
\end{enumerate}


\end{document}