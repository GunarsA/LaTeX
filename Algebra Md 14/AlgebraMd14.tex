\documentclass{article}
\usepackage{graphicx}
\graphicspath{ {./images/} }
\usepackage[export]{adjustbox}
\usepackage{amsmath}


\title{14. Gredzeni un lauki (pēdējais mājas darbs)}
\author{Gunārs Ābeltiņš}
\date{2022-06-01}

\begin{document}

\maketitle

\section*{1. Uzdevums}
\subsubsection*{Izmantojot tikai 10 lauka aksiomas un jau iepriekš pierādītās \\ teorēmas, pierādiet T19, T20.}

Teorēma:
\begin{equation*}
    \frac{a}{b}\frac{c}{d}=\frac{ab}{cd}; \frac{ac}{bc}=\frac{a}{b}
\end{equation*}
Pierādijums:
\begin{equation*}
    \frac{a}{b}\frac{c}{d}=abc^{-1}d^{-1}=(ac)(b^{-1}d^{-1})=(ac)(bd)^{-1}=\frac{ac}{bd}=\frac{ab}{cd}
\end{equation*}
\begin{equation*}
    \frac{ac}{bc}=acc^{-1}b^{-1}=(ab^{-1})(cc^{-1})=ab^{-1}=\frac{a}{b}
\end{equation*}
Teorēma:
\begin{center}
    $\frac{a}{b}=\frac{c}{d}$ tad un tikai tad, ja b un d nav 0, un $ad=bc$.
\end{center}
Pierādijums:
\begin{equation*}
    \frac{a}{b}=\frac{c}{d} \Rightarrow ab^{-1}=cd^{-1} \Rightarrow ab^{-1}db=cd^{-1}db \Rightarrow ad=bc
\end{equation*}

\section*{2. Uzdevums}
\subsubsection*{Izmantojot tikai 10 lauka aksiomas un jau iepriekš pierādītās \\ teorēmas, pierādiet T21.}

Teorēma:
\begin{equation*}
    \frac{a}{c}+\frac{a}{c}=\frac{a+b}{c};\frac{a}{b}+\frac{c}{d}=\frac{ad+bc}{bd}
\end{equation*}
Pierādijums:
\begin{equation*}
    \frac{a}{c}+\frac{a}{c}=ac^{-1}+ac^{-1}=(a+a)c^{-1}=\frac{a+b}{c}
\end{equation*}
\begin{gather*}
    \frac{a}{b}+\frac{c}{d}=ab^{-1}+cd^{-1}=ab^{-1}(dd^{-1})+cd^{-1}(bb^{-1})=\\
    =(ad)(bd)^{-1}+bc(bd)^{-1}=(ad+bc)(bd)^{-1}=\frac{ad+bc}{bd}
\end{gather*}

\section*{3. Uzdevums}
\subsubsection*{a) Uzbūvējiet summu un reizinājumu tabulu gredzenam Z6.}

Summu tabula:
\begin{center}
    \begin{tabular}{ c | c c c c c c }
          & 0 & 1 & 2 & 3 & 4 & 5 \\
        \hline
        0 & 0 & 1 & 2 & 3 & 4 & 5 \\
        1 & 1 & 2 & 3 & 4 & 5 & 0 \\
        2 & 2 & 3 & 4 & 5 & 0 & 1 \\
        3 & 3 & 4 & 5 & 0 & 1 & 2 \\
        4 & 4 & 5 & 0 & 1 & 2 & 3 \\
        5 & 5 & 0 & 1 & 2 & 3 & 4 \\
    \end{tabular}
\end{center}
Reizinājuma tabula;
\begin{center}
    \begin{tabular}{ c | c c c c c c }
          & 0 & 1 & 2 & 3 & 4 & 5 \\
        \hline
        0 & 0 & 0 & 0 & 0 & 0 & 0 \\
        1 & 0 & 1 & 2 & 3 & 4 & 5 \\
        2 & 0 & 2 & 4 & 0 & 2 & 4 \\
        3 & 0 & 3 & 0 & 3 & 0 & 3 \\
        4 & 0 & 4 & 2 & 0 & 4 & 2 \\
        5 & 0 & 5 & 4 & 3 & 2 & 1 \\
    \end{tabular}
\end{center}

\subsubsection*{b) Atrodiet Z6 visus nulles dalītājus un visus inversos elementus. Parādiet visus aprēķina soļus.}

Vērtibas tika nolasītas no tabulas.\\
Nulles dalītāji: (2, 3) un (3, 4)\\
Inversie elementi: (1, 1) un (5, 5)
\end{document}